

In this paper, we studied 5,445 bounties with a total value of \$406,425 from Bountysource along with their associated 3,509 issue reports from GitHub to study the relationship between the bounty (e.g., timing of proposing a bounty, bounty value, and bounty-usage frequency) and the issue-addressing likelihood.
We found that:
\begin{enumerate}
\item The timing of proposing bounties and the bounty-usage frequency are the most important factors that impact the issue-addressing likelihood. Issue reports for which bounties were proposed earlier are more likely to have a higher issue-addressing likelihood and a faster addressing-speed. 
\item In first-timer bounty-projects, the issue-addressing likelihood is higher for higher bounty values and in these projects, backers should consider proposing a relatively bigger bounty. 
\item Backers should be cautious when proposing small bounties on long-standing issue reports as they risk losing money without getting their issue addressed.
\end{enumerate}


Our findings suggest that backers should consider proposing a bounty early and be cautious when proposing small bounties on long-standing issue reports. Bounty platforms should allow dividing bounties between hunters, and transferring bounties to other issue reports.


