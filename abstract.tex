Due to the voluntary nature of open source software, it can be hard to find a developer to work on a particular task. For example, some issue reports may be too cumbersome and unexciting for someone to volunteer to do them, yet these issue reports may be of high priority to the success of a project. To provide an incentive for implementing such issue reports, one can propose a monetary reward, i.e., a bounty, to the developer who completes that particular task.
In this paper, we study bounties in open source projects on GitHub to better understand how bounties can be leveraged to evolve such projects in terms of addressing issue reports.

We investigated 5,445 bounties for GitHub projects. These bounties were proposed through the Bountysource platform with a total bounty value of \$406,425.
We find that 1) in general, the timing of proposing bounties and the bounty-usage frequency are the most important factors that impact the likelihood of an issue being addressed. More specifically, issue reports are more likely to be addressed if they are for projects in which bounties are used more frequently and if they are proposed earlier. 2) The bounty value that an issue report has is the most important factor that impacts the issue-addressing likelihood in the projects in which no bounties were used before. Backers in such projects proposed higher bounty values to get issues addressed. 3) There is a risk of wasting money for backers who invest money on long-standing issue reports. 
%Based on our findings, we suggest that: 1) Backers should consider proposing a bounty as early as possible and are cautious when proposing bounties on long-standing issue reports. 2) Backers of projects with no former bounty-usage experience should consider proposing higher bounty values for issue reports.
