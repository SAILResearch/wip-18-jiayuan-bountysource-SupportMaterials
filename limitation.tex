\textbf{External validity.} Threats to external validity relate to the
generalizability of our findings. In this study, we focus on
Stack Overflow, which is one of the most popular Q\&A websites for developers, hence, our results may not generalize
to other Q\&A websites. To alleviate this threat, more Q\&A
websites should be studied in the future.
We needed to conduct several qualitative analysis in
our RQs; however, it is impossible to manually study all
revisions. To minimize the bias when conducting our qualitative analysis, we took statistically representative samples
of all relevant revisions with 95\% confidence level and 5%
confidence interval~\citep{StatisticsInANutshell,conf/msr/ChenSYHGNF16}. 


\emph{Internal validity.} Threats to internal validity relate to experimenter bias and errors. Our study involved qualitative analysis in RQs. To reduce the
bias, each revision was labeled by two of the authors and
discrepancies were discussed until a consensus was reached.
We also showed that the level of inter-rate agreement of the
qualitative studies is high. Another threat of our study is caused by the way we collect the data. The accuracy of our selection criteria is 80\%, which imply that we include the noise in our quantitative study. Future study should develop more accurate method to identify the obsoleteness in Stack Overflow. 

%In addition, the conclusion drawn from our quantitative and qualitative study is based on previous obsolete information we can find on Stack Overflow. In the future the community may have different trends and patterns that we do not expect. However, our conclusion can still be useful for developers to understand better about historical obsolete knowledge on Stack Overflow. 